
\section{C / C++ 标准库中的字符串}

\subsection{C 标准库}

C 标准库是在对字符数组进行操作

\subsubsection{strlen}

\texttt{int strlen(const char *str)} :返回从 \texttt{str[0]} 开始直到 \texttt{'\textbackslash{}0'} 的字符数。注意,未开启 O2 优化时,该操作写在循环条件中复杂度是 $\Theta(N)$ 的。

\subsubsection{printf}

\texttt{printf("\%s", s)}:用 \texttt{\%s} 来输出一个字符串(字符数组)。

\subsubsection{scanf}

\texttt{scanf("\%s", s)}:用 \texttt{\%s} 来读入一个字符串(字符数组)。

\subsubsection{sscanf}

\texttt{sscanf(const char *\_\_source, const char *\_\_format, ...)}:从字符串\texttt{\_\_source}里读取变量,比如\texttt{sscanf(str,"\%d",\&a)}。

\subsubsection{sprintf}

\texttt{sprintf(char *\_\_stream, const char *\_\_format, ...)}:将\texttt{\_\_format}字符串里的内容输出到\texttt{\_\_stream}中,比如\texttt{sprintf(str,"\%d",i)}。

\subsubsection{strcmp}

\texttt{int strcmp(const char *str1, const char *str2)}:按照字典序比较 \texttt{str1 str2} 若 \texttt{str1} 字典序小返回负值, 一样返回 0 ,大返回正值 请注意,不要简单的认为只有 \texttt{0, 1, -1}  三种,在不同平台下的返回值都遵循正负,但并非都是 \texttt{0, 1, -1}

\subsubsection{strcpy}

\texttt{char *strcpy(char *str, const char *src)} : 把 \texttt{src} 中的字符复制到 \texttt{str} 中, \texttt{str} \texttt{src} 均为字符数组头指针, 返回值为 \texttt{str} 包含空终止符号 \texttt{'\textbackslash{}0'} 。

\subsubsection{strncpy}

\texttt{char *strncpy(char *str, const char *src, int cnt)} :复制至多 \texttt{cnt} 个字符到 \texttt{str} 中,若 \texttt{src} 终止而数量未达 \texttt{cnt} 则写入空字符到 \texttt{str} 直至写入总共 \texttt{cnt} 个字符。

\subsubsection{strcat}

\texttt{char *strcat(char *str1, const char *str2)} : 将 \texttt{str2} 接到 \texttt{str1} 的结尾,用 \texttt{*str2} 替换 \texttt{str1} 末尾的 \texttt{'\textbackslash{}0'}  返回 \texttt{str1} 。

\subsubsection{strstr}

\subsubsection{strchr}

\subsubsection{strrchr}

\subsection{C++ 标准库}

C++ 标准库是在对字符串对象进行操作,同时也提供对字符数组的兼容。

\subsubsection{std::string}

\begin{itemize}
\item 赋值运算符 \texttt{=} 右侧可以是 \texttt{const string / string / const char* / char*}。
\item 访问运算符 \texttt{[cur]} 返回 \texttt{cur} 位置的引用。
\item 访问函数 \texttt{data() / c\_str()} 返回一个\texttt{const char*} 指针, 内容与该 \texttt{string} 相同。
\item 容量函数 \texttt{size()} 返回字符串字符个数。
\item 还有一些其他的函数如 \texttt{find()} 找到并返回某字符位置。
\item \texttt{std :: string} 重载了比较逻辑运算符,复杂度是 $\Theta(N)$ 的。
\end{itemize}
